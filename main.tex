%%%%%%%%%%%%%%%%%
% This is an example CV created using altacv.cls (v1.1.3, 30 April 2017) written by
% LianTze Lim (liantze@gmail.com), based on the 
% Cv created by BusinessInsider at http://www.businessinsider.my/a-sample-resume-for-marissa-mayer-2016-7/?r=US&IR=T
% 
%% It may be distributed and/or modified under the
%% conditions of the LaTeX Project Public License, either version 1.3
%% of this license or (at your option) any later version.
%% The latest version of this license is in
%%    http://www.latex-project.org/lppl.txt
%% and version 1.3 or later is part of all distributions of LaTeX
%% version 2003/12/01 or later.
%%%%%%%%%%%%%%%%

%% If you want to use \orcid or the
%% academicons icons, add "academicons"
%% to the \documentclass options. 
%% Then compile with XeLaTeX or LuaLaTeX.
% \documentclass[10pt,a4paper,academicons]{altacv}

%% Use the "normalphoto" option if you want a normal photo instead of cropped to a circle
% \documentclass[10pt,a4paper,normalphoto]{altacv}

\documentclass[10pt,a4paper]{altacv}

%% AltaCV uses the fontawesome and academicon fonts
%% and packages. 
%% See texdoc.net/pkg/fontawecome and http://texdoc.net/pkg/academicons for full list of symbols.
%% When using the "academicons" option,
%% Compile with LuaLaTeX for best results. If you
%% want to use XeLaTeX, you may need to install
%% Academicons.ttf in your operating system's font %% folder.


% Change the font if you want to.

% If using pdflatex:
\usepackage[utf8]{inputenc}
\usepackage[T1]{fontenc}
\usepackage[default]{lato}
\usepackage[colorlinks=true, urlcolor=red]{hyperref}



\usepackage{eurosym}
\usepackage{amstext} % for \text
\DeclareRobustCommand{\officialeuro}{%
  \ifmmode\expandafter\text\fi
  {\fontencoding{U}\fontfamily{eurosym}\selectfont e}}

% If using xelatex or lualatex:
% \setmainfont{Lato}



% Change the colours if you want to
\definecolor{VividPurple}{HTML}{100000}
\definecolor{SlateGrey}{HTML}{2E2E2E}
\definecolor{LightGrey}{HTML}{666666}
\colorlet{heading}{VividPurple}
\colorlet{accent}{VividPurple}
\colorlet{emphasis}{SlateGrey}
\colorlet{body}{LightGrey}

% Change the bullets for itemize and rating marker
% for \cvskill if you want to
\renewcommand{\itemmarker}{{\small\textbullet}}
\renewcommand{\ratingmarker}{\faCircle}

\begin{document}

\name{Rahul Sundaresan}
\tagline{I am a graduate student currently pursuing a MS degree in Computer Science at UNC Charlotte.
My interests lie in computer networks and web development with React. I am looking for remote software development opportunities for Fall 2020}


\personalinfo{%
  % Not all of these are required!
  % You can add your own with \printinfo{symbol}{detail}
    \email{\href{mailto:rahul@rsun.dev}{rahul@rsun.dev}}
    \phone{\href{tel:+19802550390}{(980)-255-0390}}
    \homepage{\url{https://rsun.dev}}
    \linkedin{\url{https://www.linkedin.com/in/rahul-sundaresan/}}
  % I'm just making this up though.
%   \orcid{orcid.org/0000-0000-0000-0000} % Obviously making this up too. If you want to use this field (and also other academicons symbols), add "academicons" option to \documentclass{altacv}
}

%% Make the header extend all the way to the right, if you want.


%% Provide the file name containing the sidebar contents as an optional parameter to \cvsection.
%% You can always just use \marginpar{...} if you do
%% not need to align the top of the contents to any
%% \cvsection title in the "main" bar.
\begin{twocolumn}[
    
    \makecvheader
 
]


\cvsection{Education}

\cvevent{M.S. Computer Science \hfill GPA: 3.6/4.0}{University of North Carolina at Charlotte}{2019--Current}{Charlotte, North Carolina}

\divider

\cvevent{B.Tech Computer Science \hfill GPA: 6.9/10.0}{SASTRA University}{2015--2019}{Thanjavur, India}

    
\cvsection{Professional Experience}

\cvevent{Product Consultant}{Juniper Networks}{September 2021--Current}{Hartford, Connecticut}
\begin{itemize}
    \item Scripted a dynamic inventory plugin for Ansible using Paragon Automation Suite as the source of truth
    \item Implemented an automated provisioning solution using flask (Python) and Kea DHCP server to provision hundreds of Juniper devices based on a single souce of truth. This led to a 1500\% decrease in required engineering resources and a 400\% decrease in provisioning time.
    \item Stack: \cvtag{Python} \cvtag{Flask} \cvtag{Kea DHCP server}
\end{itemize}

\divider

\cvevent{Full Stack developer}{ING}{March 2021--December 2021}{San Francisco, California}
\begin{itemize}
    \item Implemented features in the company's flagship app using React Native
    \item Built REST APIs using Google Cloud functions which are consumed by the app
    \item Architected sequence diagrams for the design of the app using Miro
    \item Built a location based search engine using a custom built implementation of Geohashes and programmed a search filter using Dice's Coefficient
    \item Wrote Unit tests using Mocha to verify and validate functionality
    \item Stack: \cvtag{React Native} \cvtag{Firebase} \cvtag{Google Cloud functions} \cvtag{Node.js} \cvtag{JavaScript} \cvtag{Miro}
\end{itemize}
\divider

\cvevent{Network and System Administrator (Freelance)}{Pat \& Venky}{January 2018--Current}{Chennai, India}
\begin{itemize}
    \item Decommissioned an aging Windows based fileshare system and deployed a high availability NAS setup which drastically improved uptime and provided disaster recovery options.
    \item Configured fileshares with access based controls preventing unauthorized access to proprietary business data.
    \item Set up separate VLANs, creating a more secure network.
    \item Set up a cloud hosted Unifi controller on google cloud platform (GCP) to allow for centralized deployment of future sites and reduce CAPEX costs.
    \item Configured appropriate firewall policies for the VM to prevent unauthorized access.
\end{itemize}
\divider


\cvevent{Research Assistant}{University of Nevada at Las Vegas}{June 2020--August 2020}{Las Vegas, Nevada}
\begin{itemize}
    \item Collaborated with Dr. Yoohwan Kim and Dr. Ju-Yeon Jo in researching the effectiveness of Multipath TCP in improving TCP throughput.
    \item Deployed network simulations to measure the throughput under varying latency and loss rate for Singlepath and Multipath TCP networks.
    \item Processed the output of iperf3's JSON results and graphed the data to visualize trends in throughput.
    \item Stack: \cvtag{Python} \cvtag{Mininet} \cvtag{Plotly} \cvtag{pandas}
\end{itemize}
\divider




\cvsection{Projects}

\href{https://rsun.dev}{Website Portfolio}
\begin{itemize}
    \item Built a personal website using Gatsby with CI/CD on Netlify.
    \item The PDF resume on the site is written in \LaTeX{} and generated with GitHub actions
    \item The website automatically integrates the latest version of the resume whenever a commit is made to the \LaTeX{} repository.
\end{itemize}
    
\href{https://github.com/ngrover2/teammaking}{Teammaking}
\begin{itemize}
    \item Developed a full stack application that generates teams based on a survey.
    \item Incrementally built using React, Node and using mySQL following an agile development methodology.
\end{itemize}


\href{https://github.com/networking-discord/network-ranger}{Network ranger}
\begin{itemize}
    \item Refactored the code base and code reviewed a Discord bot that manages user roles and permissions in a discord server.
    \item Optimized  a docker workflow to prevent a security breach.
\end{itemize}
    
\href{https://github.com/rahul-sundaresan/network-gui/}{Network-gui}
\begin{itemize}
    \item Created a React app queries a SDN switch and uses react-d3-graph to graph the topology of the network.
\end{itemize}


\href{https://github.com/rahul-sundaresan/ccn-6166-p2}{TCP over UDP}
\begin{itemize}
    \item Implemented the TCP go-back-N and selective repeat protocols over UDP.
    \item  A toy project written in python to better understand the implementation of TCP.
\end{itemize}


\href{https://github.com/rahul-sundaresan/6166-project-1}{Tiny HTTP system}
\begin{itemize}
    \item Implemented a very basic HTTP client and server in less than 100 LoC combined which supports PUT and POST requests with file transfer.
\end{itemize}


\href{https://github.com/rahul-sundaresan/agency-spotter-scraper}{Agency spotter scraper}
\begin{itemize}
    \item Designed a pipeline to scrape the agency spotter website using Python's Beautiful Soup library and a bash script.
\end{itemize}



\cvsection{Skills}

\cvsubsection{Programming}
    \cvtag{LaTeX}
    \cvtag{Python 3}
    \cvtag{Java (Android Development)}
    \cvtag{React}
    \cvtag{Gatsby}
    \cvtag{\LaTeX}
    
\divider

\cvsubsection{Tools and technologies}
    \cvtag{Git}
    \cvtag{REST APIs}
    \cvtag{IPv6}
    \cvtag{Software Defined Networks (SDN)}
    \cvtag{Mininet}
    \cvtag{Netlify}
\end{twocolumn}

\end{document}
